\documentclass[12pt, a4paper, oneside]{ctexart}
\usepackage{amsmath, amsthm, amssymb, appendix, bm, graphicx, hyperref, mathrsfs, geometry}
\usepackage{float}
\usepackage{subcaption}
\usepackage{listings}
\usepackage{longtable}
\usepackage[dvipsnames]{xcolor}
\usepackage{subfiles}
\usepackage{fontspec}
\usepackage{array}

\linespread{1.5}
\pagestyle{plain}
\geometry{a4paper, scale=0.8}

% 定义书写语法时的listings style
\lstdefinestyle{grammar}{
    basicstyle=\ttfamily,
    breaklines=true,
    mathescape=true,
    morekeywords={ProgramStart, ProgramStruct, ProgramHead, ProgramBody, IdList, ConstDeclarations, ConstDeclaration, ConstValue, VarDeclarations, VarDeclaration, Type, BasicType, BasicType, Period, SubprogramDeclarations, Subprogram, SubprogramHead, FormalParameter, ParameterList, Parameter, VarParameter, ValueParameter, SubprogramBody, CompoundStatement, StatementList, Statement, Variable, IdVarPart, ProcedureCall, ElsePart, ExpressionList, Expression, SimpleExpression, Term, Factor, AddOperator, MultiplyOperator, RelationOperator},
    keywordstyle=\bfseries\color{NavyBlue},
    emphstyle={\bfseries\color{Rhodamine}},
    emph={program, id, num, true, false,var, array, of, integer, real, boolean, char, digits, procedure, function, begin, end, assignOp, if, then, for, to ,do ,while, else, not, or, div, mod, and, const, letter}
}

% 定义书写C#时的listings style
\lstdefinestyle{csharp}{
    language=[sharp]c,
    breaklines=true,
    basicstyle=\ttfamily,
    keywordstyle=\bfseries\color{violet},
    emphstyle=\bfseries\color{blue},
    morekeywords={required, get, set, init},
    showstringspaces=false,
}

% 定义书写C时的listings style
\lstdefinestyle{c}{
    language=c,
    breaklines=true,
    basicstyle=\ttfamily,
    keywordstyle=\bfseries\color{blue},
    showstringspaces=false,
}

% 定义书写Pascal时的listings style
\lstdefinestyle{pascal}{
    language=Pascal,
    breaklines=true,
    basicstyle=\ttfamily,
    keywordstyle=\bfseries\color{violet},
    emphstyle=\bfseries\color{blue},
}

\begin{document}

\begin{titlepage}
% 标题
\begin{center}
    \Huge{\textbf{北\quad 京\quad 邮\quad 电\quad 大\quad 学}}

    \vspace{2em}
    
    \Large{\textbf{《编译原理与技术课程设计》}}
    
    \Large{\textbf{报\qquad 告}}

    \vspace{4em}

    \large{指导教师: \underline{\makebox[8em][c]{王雅文}}}
\end{center}

\vspace{6em}

% 个人信息
\begin{table}[h]
    \centering
    \begin{tabular}{|c|c|c|c|}
        \hline
        \textbf{姓名} & \textbf{班级} & \textbf{学号} & \textbf{备注} \\
        \hline
        张弈纶 & 2021211308 & 2021211177 & \\
        \hline
        兰建国 & 2021211308 & 2021211179 & \\
        \hline
        任昌骏 & 2021211308 & 2021211180 & 组长 \\
        \hline
        肖可扬 & 2021211308 & 2021211186 & \\
        \hline
        杜含韵 & 2021211308 & 2021211188 & \\
        \hline
        陈劲淞 & 2020219308 & 2018211608 & \\
        \hline
    \end{tabular}
\end{table}

% 封底
\vspace{8em}

\begin{center}
    \Large{\textbf{计算机学院(国家示范性软件学院)}}

    \Large{2024年5月}
\end{center}
\end{titlepage}

\clearpage

% 目录
% 目录的页码和正文的页码不一致
\pagenumbering{Roman}
\setcounter{page}{1}
\tableofcontents

\clearpage
\setcounter{page}{1}
\pagenumbering{arabic}

\begin{center}
    \Large{\textbf{Pascal-S 语言编译程序的设计与实现}}
\end{center}

\subfile{contents/tasks-and-objectives}

\subfile{contents/requirements-analysis}

\subfile{contents/general-design}

\subfile{contents/detailed-design}

\subfile{contents/source}

\subfile{contents/program-test}

\subfile{contents/summary}

\clearpage

\bibliographystyle{unsrt}
\bibliography{ref}

\end{document}
